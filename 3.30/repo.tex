%!Mode:: "TeX:UTF-8"
\documentclass[a4paper,11pt,UTF8]{ctexart}
\usepackage{indentfirst} %缩进
\usepackage{xeCJK}    %使用系统字体
\usepackage{fancyhdr} %自定义页眉页脚
\pagestyle{empty}                   %不设置页眉页脚
\usepackage{amsmath, amsthm, amssymb, amsfonts} %数学公式
\usepackage[a4paper,left=3cm,right=3cm,top=3cm,bottom=3cm]{geometry}
%\usepackage[tmargin=1in,bmargin=1in,lmargin=1.25in,rmargin=1.25in]{geometry}.
\usepackage{booktabs} %插入表格
\usepackage[section]{placeins} %避免浮动
\usepackage{listings} %插入代码
\usepackage{ctex}     %中文宏包
\usepackage[svgnames, table]{xcolor} %彩色表格
\usepackage{algorithm}          %伪代码
\usepackage{algorithmicx}
\usepackage{algpseudocode}
\usepackage{algorithm,algpseudocode,float}
\usepackage{lipsum}
\usepackage{enumitem}           %调整列举环境
\usepackage{url}
\usepackage{fontspec,xunicode}
\usepackage{cite}
\defaultfontfeatures{Mapping=tex-text} %如果没有它,会有一些 tex 特殊字符无法正常使用,比如连字符。

\usepackage{graphicx}
\usepackage{subfigure}
\graphicspath{{imgs/}}

%%%%%%%%%%%%%%%%%%%%%%%%%%%%%%%%%%%%%%%%%%%%%%%%%%%%%%%%%%%%%%%%
% 缩进及行间距
%%%%%%%%%%%%%%%%%%%%%%%%%%%%%%%%%%%%%%%%%%%%%%%%%%%%%%%%%%%%%%%%
\setlength{\parindent}{22pt} %重新定义缩进长度
\setlength{\baselineskip}{20pt}  %定义行间距
%\renewcommand{\baselinestretch}{1.1} %定义行间距

%%%%%%%%%%%%%%%%%%%%%%%%%%%%%%%%%%%%%%%%%%%%%%%%%%%%%%%%%%%%%%%%
% 列表设置
%%%%%%%%%%%%%%%%%%%%%%%%%%%%%%%%%%%%%%%%%%%%%%%%%%%%%%%%%%%%%%%%
\setenumerate{fullwidth,itemindent=\parindent,listparindent=\parindent,itemsep=0ex,partopsep=0pt,parsep=0ex}
\setenumerate[2]{label=\alph*),leftmargin=1.5em}  %二级item设置
\setitemize{itemindent=38pt,leftmargin=0pt,itemsep=-0.4ex,listparindent=26pt,partopsep=0pt,parsep=0.5ex,topsep=-0.25ex}
\setdescription{itemindent=38pt,leftmargin=0pt,itemsep=-0.4ex,listparindent=26pt,partopsep=0pt,parsep=0.5ex,topsep=-0.25ex}

%%%%%%%%%%%%%%%%%%%%%%%%%%%%%%%%%%%%%%%%%%%%%%%%%%%%%%%%%%%%%%%%
% 图的标题行间距设置
%%%%%%%%%%%%%%%%%%%%%%%%%%%%%%%%%%%%%%%%%%%%%%%%%%%%%%%%%%%%%%%%
\newcommand{\bottomcaption}{%
\setlength{\abovecaptionskip}{6pt}%
\setlength{\belowcaptionskip}{6pt}%
\caption}


%%%%%%%%%%%%%%%%%%%%%%%%%%%%%%%%%%%%%%%%%%%%%%%%%%%%%%%%%%%%%%%%
% 字体定义
%%%%%%%%%%%%%%%%%%%%%%%%%%%%%%%%%%%%%%%%%%%%%%%%%%%%%%%%%%%%%%%%
\setmainfont{Times New Roman}  %默认英文字体.serif是有衬线字体sans serif无衬线字体
\setmonofont{Consolas}
\setCJKmainfont[ItalicFont={楷体}, BoldFont={黑体}]{宋体}%衬线字体 缺省中文字体为
\setCJKsansfont{黑体}
\punctstyle{hangmobanjiao}
%-----------------------xeCJK下设置中文字体------------------------------%
\setCJKfamilyfont{song}{SimSun}                             %宋体 song
\newcommand{\song}{\CJKfamily{song}}
\setCJKfamilyfont{fs}{FangSong}                      %仿宋  fs
\newcommand{\fs}{\CJKfamily{fs}}
\setCJKfamilyfont{ktgb}{KaiTi}                      %楷体2312 ktgb
\newcommand{\ktgb}{\CJKfamily{ktgb}}
\setCJKfamilyfont{yh}{Microsoft YaHei}                    %微软雅黑 yh
\newcommand{\yh}{\CJKfamily{yh}}
\setCJKfamilyfont{hei}{SimHei}                              %黑体  hei
\newcommand{\hei}{\CJKfamily{hei}}
\setCJKfamilyfont{hwxk}{STXingkai}                                %华文行楷  hwxk
\newcommand{\hwxk}{\CJKfamily{hwxk}}
%------------------------------设置字体大小------------------------%
\newcommand{\shiyanbaogao}{\fontsize{36pt}{\baselineskip}\selectfont}
\newcommand{\chuhao}{\fontsize{42pt}{\baselineskip}\selectfont}     %初号
\newcommand{\xiaochuhao}{\fontsize{36pt}{\baselineskip}\selectfont} %小初号
\newcommand{\yihao}{\fontsize{28pt}{\baselineskip}\selectfont}      %一号
\newcommand{\erhao}{\fontsize{21pt}{\baselineskip}\selectfont}      %二号
\newcommand{\xiaoerhao}{\fontsize{18pt}{\baselineskip}\selectfont}  %小二号
\newcommand{\sanhao}{\fontsize{15.75pt}{\baselineskip}\selectfont}  %三号
\newcommand{\sihao}{\fontsize{14pt}{\baselineskip}\selectfont}       %四号
\newcommand{\xiaosihao}{\fontsize{12pt}{\baselineskip}\selectfont}  %小四号
\newcommand{\wuhao}{\fontsize{10.5pt}{\baselineskip}\selectfont}    %五号
\newcommand{\xiaowuhao}{\fontsize{9pt}{\baselineskip}\selectfont}   %小五号
\newcommand{\liuhao}{\fontsize{7.875pt}{\baselineskip}\selectfont}  %六号
\newcommand{\qihao}{\fontsize{5.25pt}{\baselineskip}\selectfont}    %七号

%%%%%%%%%%%%%%%%%%%%%%%%%%%%%%%%%%%%%%%%%%%%%%%%%%%%%%%%%%%%%%%%
% 图题字体大小相同
%%%%%%%%%%%%%%%%%%%%%%%%%%%%%%%%%%%%%%%%%%%%%%%%%%%%%%%%%%%%%%%%
\usepackage{caption}
\captionsetup{font={footnotesize}}   % footnotesize = 9pt
\captionsetup[lstlisting]{font={footnotesize}}

%%%%%%%%%%%%%%%%%%%%%%%%%%%%%%%%%%%%%%%%%%%%%%%%%%%%%%%%%%%%%%%%
% 重定义枚举编号为 1),2)...
%%%%%%%%%%%%%%%%%%%%%%%%%%%%%%%%%%%%%%%%%%%%%%%%%%%%%%%%%%%%%%%%
\renewcommand{\labelenumi}{\theenumi)}


%%%%%%%%%%%%%%%%%%%%%%%%%%%%%%%%%%%%%%%%%%%%%%%%%%%%%%%%%%%%%%%%
% 重定义section标题
%%%%%%%%%%%%%%%%%%%%%%%%%%%%%%%%%%%%%%%%%%%%%%%%%%%%%%%%%%%%%%%%
\CTEXsetup[format={\sihao\CJKfamily{zhhei}\zihao{4}},number={\chinese{section}},name={,、~},aftername={},indent={0pt},beforeskip={6pt},afterskip={6pt},format+={\flushleft}]{section}
\CTEXsetup[format={\Large\bfseries\CJKfamily{zhkai}\zihao{5}},name={(,)},number={\chinese{subsection}},aftername={},indent={22pt},beforeskip={14pt},afterskip={2pt}]{subsection}
\CTEXsetup[number={\chinese{section}},name={附录, ~~ }]{appendix}



%%%%%%%%%%%%%%%%%%%%%%%%%%%%%%%%%%%%%%%%%%%%%%%%%%%%%%%%%%%%%%%%
% 标题名称中文化
%%%%%%%%%%%%%%%%%%%%%%%%%%%%%%%%%%%%%%%%%%%%%%%%%%%%%%%%%%%%%%%%
\renewcommand\figurename{\hei 图}
\renewcommand\tablename{\hei 表}
\renewcommand\lstlistingname{\hei 代码}
\renewcommand{\algorithmicrequire}{\textbf{输入:}}
\renewcommand{\algorithmicensure}{\textbf{输出:}}
\newtheorem{define}{定义}

%%%%%%%%%%%%%%%%%%%%%%%%%%%%%%%%%%%%%%%%%%%%%%%%%%%%%%%%%%%%%%%%
% 代码设置
%%%%%%%%%%%%%%%%%%%%%%%%%%%%%%%%%%%%%%%%%%%%%%%%%%%%%%%%%%%%%%%%
\lstset{
 columns=fixed,
 numbers=left,                                        % 在左侧显示行号
 numberstyle=\tiny\color{gray},                       % 设定行号格式
 frame=single,                                        % 单线背景边框
 breaklines=true,                                     % 设定LaTeX对过长的代码行进行自动换行
 keywordstyle=\color[RGB]{40,40,255},                 % 设定关键字颜色
 numberstyle=\footnotesize\color{darkgray},
 commentstyle=\it\color[RGB]{0,96,96},                % 设置代码注释的格式
 stringstyle=\rmfamily\slshape\color[RGB]{128,0,0},   % 设置字符串格式
 showstringspaces=false,                              % 不显示字符串中的空格
 language=java,                                        % 设置语言
 basicstyle=\linespread{1.0}\xiaowuhao\ttfamily,                      % 字体字号
 %lineskip=10pt,
 %baselinestretch=1,
}

%%%%%%%%%%%%%%%%%%%%%%%%%%%%%%%%%%%%%%%%%%%%%%%%%%%%%%%%%%%%%%%%
% 伪代码分页
%%%%%%%%%%%%%%%%%%%%%%%%%%%%%%%%%%%%%%%%%%%%%%%%%%%%%%%%%%%%%%%%
\makeatletter
\renewcommand{\ALG@name}{算法}
\newenvironment{breakablealgorithm}
  {% \begin{breakablealgorithm}
   \begin{center}
     \refstepcounter{algorithm}% New algorithm
     \hrule height.8pt depth0pt \kern2pt% \@fs@pre for \@fs@ruled
     \renewcommand{\caption}[2][\relax]{% Make a new \caption
       {\raggedright\textbf{\ALG@name~\thealgorithm} ##2\par}%
       \ifx\relax##1\relax % #1 is \relax
         \addcontentsline{loa}{algorithm}{\protect\numberline{\thealgorithm}##2}%
       \else % #1 is not \relax
         \addcontentsline{loa}{algorithm}{\protect\numberline{\thealgorithm}##1}%
       \fi
       \kern2pt\hrule\kern2pt
     }
  }{% \end{breakablealgorithm}
     \kern2pt\hrule\relax% \@fs@post for \@fs@ruled
   \end{center}
  }
\makeatother



\begin{document}
\xiaosihao\song

\begin{titlepage}
\center{\yihao{\ktgb{中山大学数据科学与计算机学院\\操作系统实验课程}}}
\vspace{1cm}


\center{\shiyanbaogao{\ktgb{实~验~报~告}}}
\vspace{2cm}

\begin{center}
\begin{large}
\begin{tabular}{rc}
\xiaoerhao{\hei{教\qquad 师}}& \sanhao{\hei{凌应标}}\\
\cline{2-2}\\
\xiaoerhao{\hei{学\qquad 号}}& \hspace{1.7cm}\sanhao{\hei{17341035\hspace{1.7cm}}} \\
\cline{2-2}\\
\xiaoerhao{\hei{姓\qquad 名}}& \sanhao{\hei{傅畅}}\\
\cline{2-2}\\
\xiaoerhao{\hei{实验名称}}& \sanhao{\hei{实验四()}}\\
\cline{2-2}\\

\end{tabular}
\end{large}
\end{center}
\vfill \hfill
\end{titlepage}
\clearpage

\centerline{\\[10pt]\erhao{\fs{实~验~一}}}
\centerline{\\[10pt]\yihao{\fs{接~管~裸~机~控~制~权}}}

\leftline{\\[8pt]\xiaosihao{\hei{\hspace{1.5em} 姓名:傅畅\hfill 学号:17341038 \hfill  }}}

\leftline{\\[8pt]\xiaosihao{\hei{\hspace{1.5em} 邮箱:fuch8@mail2.sysu.edu.cn \hfill}}}

\leftline{\\[8pt]\xiaosihao{\hei{\hspace{1.5em} 实验时间:周五(3-4节) \hfill }}}


\setcounter{tocdepth}{2}
\setlength{\parskip}{0pt}
\tableofcontents
\clearpage


\section{实验要求}

	\subsection{利用时钟中断,在右下角轮流显示转轮}

	操作系统工作期间,利用时钟中断,在屏幕24行79列位置轮流显示’|’、’/’和'$\backslash$',适当控制显示速度,以方便观察效果。

	\subsection{编写键盘中断响应程序}
	编写键盘中断响应程序,原有的你设计的用户程序运行时,键盘事件会做出有事反应:当键盘有按键时,屏幕适当位置显示”OUCH! OUCH!”。
	\subsection{编写软中断服务程序}
	在内核中,对33号、34号、35号和36号中断编写中断服务程序,分别在屏幕1/4区域内显示一些个性化信息。再编写一个汇编语言的程序,作为用户程序,利用int 33、int 34、int 35和int 36产生中断调用你这4个服务程序。

\section{实验配置}

	\subsection{实验支撑环境}
		\begin{itemize} 
			\item 硬件:个人计算机
			\item 主机操作系统:Linux 4.18.0-16-generic 17-Ubuntu
			\item 虚拟机软件: Bochs 2.6.9
		\end{itemize}
	
	
	 
\section{x86保护模式学习}
	\subsection{使用选择子访存}
		
	\subsection{分页机制}
	\subsection{中断选择子}

	% \lstset{language=={[x86nasm]Assembler}}
	% \begin{lstlisting}[caption={},tabsize=4,basicstyle=\footnotesize,captionpos=b]
	
	% \end{lstlisting}
	

\section{实验代码设计}
	\subsection{从软盘到使用硬盘启动}
	由于使用空间比较大,从实验四开始,我决定使用硬盘作为主要存储方式,bochs的配置修改如下
	\lstset{language=={[x86nasm]Assembler}}
	\begin{lstlisting}[caption={bochs硬盘配置参数},tabsize=4,basicstyle=\footnotesize,captionpos=b]
ata0-master: type=disk, mode=flat, translation=auto, path="h.img", cylinders=2, heads=16, spt=64, biosdetect=auto, model="Generic 1234"

boot:disk	
	\end{lstlisting}

	为了方便地制作硬盘镜像,我编写了批处理写二进制文件的脚本,
	\lstset{language=={[x86nasm]Assembler}}
	\begin{lstlisting}[caption={make diskimg.sh},tabsize=4,basicstyle=\footnotesize,captionpos=b]
fil=h.img
make
cat cmbr.bin > $fil # clear and cat in

cat ccore.bin >> $fil
len=$[10*512-$(stat -c %s "$fil")]
dd if=/dev/zero count=$len bs=1 | cat>> $fil

cat c1.bin >> $fil
len=$[2*16*64*512-$(stat -c %s "$fil")]
dd if=/dev/zero count=$len bs=1 | cat>> $fil
	\end{lstlisting}

	\subsection{mbr.asm}
		\subsubsection{加载内核}
			\begin{enumerate}
				\item 读取一个硬盘扇区,到目的物理地址
				\item 根据内核程序头来确定
			\end{enumerate}

			\lstset{language=={[x86nasm]Assembler}}
			\begin{lstlisting}[caption={Load\_progam},tabsize=4,basicstyle=\footnotesize,captionpos=b]
Load_program:;以下加载程序,
			;栈中的第一个参数为被加载程序的目的物理地址,第二个参数为程序在硬盘中的起始扇区
	pushad
			mov edi,[esp+40]

			mov eax,[esp+36]
			mov ebx,edi                        ;起始地址
			call read_hard_disk_0              ;以下读取程序的起始部分(一个扇区)

			;以下判断整个程序有多大
			mov eax,[edi]                      ;核心程序尺寸
			xor edx,edx
			mov ecx,512                        ;512字节每扇区
			div ecx

			or edx,edx
			jnz @1                             ;未除尽,因此结果比实际扇区数少1
			dec eax                            ;已经读了一个扇区,扇区总数减1
	@1:
			or eax,eax                         ;考虑实际长度≤512个字节的情况
			jz endLoad_program                             ;EAX=0 ?

			;读取剩余的扇区
			mov ecx,eax                        ;32位模式下的LOOP使用ECX
			mov eax,[esp+36]
			inc eax                            ;从下一个逻辑扇区接着读
	@2:
			call read_hard_disk_0
			inc eax
			loop @2                            ;循环读,直到读完整个内核

		endLoad_program:
		popad
		ret
				
			\end{lstlisting}
			其中,read\_hard\_disk将逻辑扇区号的一个扇区加载到目标地址
			\lstset{language=={[x86nasm]Assembler}}
		\begin{lstlisting}[caption={read\_hard\_disk\_0},tabsize=4,basicstyle=\footnotesize,captionpos=b]
read_hard_disk_0:                           
			;从硬盘读取一个逻辑扇区
			;EAX=逻辑扇区号
			;DS:EBX=目标缓冲区地址
			;返回:EBX=EBX+512 
	push eax 
	push ecx
	push edx

	push eax

	mov dx,0x1f2
	mov al,1
	out dx,al                          ;读取的扇区数

	inc dx                             ;0x1f3
	pop eax
	out dx,al                          ;LBA地址7~0

	inc dx                             ;0x1f4
	mov cl,8
	shr eax,cl
	out dx,al                          ;LBA地址15~8

	inc dx                             ;0x1f5
	shr eax,cl
	out dx,al                          ;LBA地址23~16

	inc dx                             ;0x1f6
	shr eax,cl
	or al,0xe0                         ;第一硬盘  LBA地址27~24
	out dx,al

	inc dx                             ;0x1f7
	mov al,0x20                        ;读命令
	out dx,al

	.waits:
	in al,dx
	and al,0x88
	cmp al,0x08
	jnz .waits                         ;不忙,且硬盘已准备好数据传输 

	mov ecx,256                        ;总共要读取的字数
	mov dx,0x1f0
	.readw:
	in ax,dx
	mov [ebx],ax
	add ebx,2
	loop .readw

	pop edx
	pop ecx
	pop eax

	ret
		\end{lstlisting}

			
		\subsubsection{安装GDT}
			\begin{enumerate}
				\item 代码段和栈段和选择子
				

				\item lgdt,sgdt, gdtr
			\end{enumerate}
			
\lstset{language=={[x86nasm]Assembler}}
	\begin{lstlisting}[caption={make gdt},tabsize=4,basicstyle=\footnotesize,captionpos=b]
		;计算GDT所在的逻辑段地址
		mov eax,[cs:pgdt+0x02]             ;GDT的32位物理地址 
		xor edx,edx
		mov ebx,16
		div ebx                            ;分解成16位逻辑地址 

		mov ds,eax                         ;令DS指向该段以进行操作
		mov ebx,edx                        ;段内起始偏移地址 

		;跳过0#号描述符的槽位 
		;创建1#描述符,保护模式下的代码段描述符
		mov dword [ebx+0x08],0x0000ffff    ;基地址为0,界限0xFFFFF,DPL=00 
		mov dword [ebx+0x0c],0x00cf9800    ;4KB粒度,代码段描述符,向上扩展 

		;创建2#描述符,保护模式下的数据段和堆栈段描述符 
		mov dword [ebx+0x10],0x0000ffff    ;基地址为0,界限0xFFFFF,DPL=00
		mov dword [ebx+0x14],0x00cf9200    ;4KB粒度,数据段描述符,向上扩展 

		;初始化描述符表寄存器GDTR
         mov word [cs: pgdt],23             ;描述符表的界限   
 
         lgdt [cs: pgdt]
      
         in al,0x92                         ;南桥芯片内的端口 
         or al,0000_0010B
         out 0x92,al                        ;打开A20

         cli                                ;中断机制尚未工作

         mov eax,cr0                  
         or eax,1
         mov cr0,eax                        ;设置PE位
      
         ;以下进入保护模式... ...
         jmp dword 0x0008:flush             ;16位的描述符选择子:32位偏移
                                            ;清流水线并串行化处理器
	\end{lstlisting}

			
		\subsubsection{开启分页}
			\begin{enumerate}
				\item 一张目录和一张页表
				\item 初始恒等映射
				\item 手动修改部分寄存器
			\end{enumerate}
\lstset{language=={[x86nasm]Assembler}}
	\begin{lstlisting}[caption={手动完善页目录和页表},tabsize=4,basicstyle=\footnotesize,captionpos=b]
		pge:
		;准备打开分页机制.
			 
		;创建系统内核的页目录表PDT
		mov ebx,0x00020000                 ;页目录表PDT的物理地址
		
		;在页目录内创建指向页目录表自己的目录项
		mov dword [ebx+4092],0x00020003 

		mov edx,0x00021003                 ;MBR空间有限,后面尽量不使用立即数
		;在页目录内创建与线性地址0x00000000对应的目录项
		mov [ebx+0x000],edx                ;写入目录项(页表的物理地址和属性)      
										   ;此目录项仅用于过渡。
		;在页目录内创建与线性地址0x80000000对应的目录项
		mov [ebx+0x800],edx                ;写入目录项(页表的物理地址和属性)

		;创建与上面那个目录项相对应的页表,初始化页表项 
		mov ebx,0x00021000                 ;页表的物理地址
		xor eax,eax                        ;起始页的物理地址 
		xor esi,esi
 .b1:       
		mov edx,eax
		or edx,0x00000003                                                      
		mov [ebx+esi*4],edx                ;登记页的物理地址
		add eax,0x1000                     ;下一个相邻页的物理地址 
		inc esi
		cmp esi,256                        ;仅低端1MB内存对应的页才是有效的 
		jl .b1
	\end{lstlisting}

\lstset{language=={[x86nasm]Assembler}}
	\begin{lstlisting}[caption={start paging},tabsize=4,basicstyle=\footnotesize,captionpos=b]
		;令CR3寄存器指向页目录,并正式开启页功能 
		mov eax,0x00020000                 ;PCD=PWT=0
		mov cr3,eax

		;将GDT的线性地址映射到从0x80000000开始的相同位置 
		sgdt [pgdt]
		mov ebx,[pgdt+2]
		add dword [pgdt+2],0x80000000      ;GDTR也用的是线性地址
		lgdt [pgdt]

		mov eax,cr0
		or eax,0x80000000
		mov cr0,eax                        ;开启分页机制
	\end{lstlisting}

% \lstset{language=={[x86nasm]Assembler}}
% 	\begin{lstlisting}[caption={bochsrc},tabsize=4,basicstyle=\footnotesize,captionpos=b]
% 	\end{lstlisting}

% \lstset{language=={[x86nasm]Assembler}}
% 	\begin{lstlisting}[caption={bochsrc},tabsize=4,basicstyle=\footnotesize,captionpos=b]
% 	\end{lstlisting}

	\subsection{core.asm}
		\subsubsection{安装idt}
		\lstset{language=={[x86nasm]Assembler}}
	\begin{lstlisting}[caption={安装 IDT},tabsize=4,basicstyle=\footnotesize,captionpos=b]
		mov eax,general_exception_handler  ;门代码在段内偏移地址
		mov bx,flat_4gb_code_seg_sel       ;门代码所在段的选择子
		mov cx,0x8e00                      ;32位中断门,0特权级
		call flat_4gb_code_seg_sel:make_gate_descriptor

		mov ebx,idt_linear_address         ;中断描述符表的线性地址
		xor esi,esi
 .idt0:
		mov [ebx+esi*8],eax
		mov [ebx+esi*8+4],edx
		inc esi
		cmp esi,19                         ;安装前20个异常中断处理过程
		jle .idt0

		;其余为保留或硬件使用的中断向量
		mov eax,general_interrupt_handler  ;门代码在段内偏移地址
		mov bx,flat_4gb_code_seg_sel       ;门代码所在段的选择子
		mov cx,0x8e00                      ;32位中断门,0特权级
		call flat_4gb_code_seg_sel:make_gate_descriptor

		mov ebx,idt_linear_address         ;中断描述符表的线性地址
 .idt1:
		mov [ebx+esi*8],eax
		mov [ebx+esi*8+4],edx
		inc esi
		cmp esi,255                        ;安装普通的中断处理过程
		jle .idt1

		;设置实时时钟中断处理过程
		mov eax,rtm_0x70_interrupt_handle  ;门代码在段内偏移地址
		mov bx,flat_4gb_code_seg_sel       ;门代码所在段的选择子
		mov cx,0x8e00                      ;32位中断门,0特权级
		call flat_4gb_code_seg_sel:make_gate_descriptor

		mov ebx,idt_linear_address         ;中断描述符表的线性地址
		mov [ebx+0x70*8],eax
		mov [ebx+0x70*8+4],edx

	   ; set the keyboard interruption
		mov eax, keyboard_interrupt_handle
		mov bx, flat_4gb_code_seg_sel
		mov cx, 0x8e00
		call flat_4gb_code_seg_sel:make_gate_descriptor

		mov ebx, idt_linear_address
		mov [ebx+0x21*8], eax
		mov [ebx+0x21*8+4], edx

		; set personal interrupt1
		mov eax, personal1_interrupt_handle
		mov bx, flat_4gb_code_seg_sel
		mov cx, 0x8e00
		call flat_4gb_code_seg_sel:make_gate_descriptor

		mov ebx, idt_linear_address
		mov [ebx+0x11*8], eax
		mov [ebx+0x11*8+4] , edx

		; set personal interrupt2
		mov eax, personal2_interrupt_handle
		mov bx, flat_4gb_code_seg_sel
		mov cx, 0x8e00
		call flat_4gb_code_seg_sel:make_gate_descriptor

		mov ebx, idt_linear_address
		mov [ebx+0x12*8], eax
		mov [ebx+0x12*8+4] , edx
		; set personal interrupt3
		mov eax, personal3_interrupt_handle
		mov bx, flat_4gb_code_seg_sel
		mov cx, 0x8e00
		call flat_4gb_code_seg_sel:make_gate_descriptor

		mov ebx, idt_linear_address
		mov [ebx+0x13*8], eax
		mov [ebx+0x13*8+4] , edx
		; set personal interrupt4
		mov eax, personal4_interrupt_handle
		mov bx, flat_4gb_code_seg_sel
		mov cx, 0x8e00
		call flat_4gb_code_seg_sel:make_gate_descriptor

		mov ebx, idt_linear_address
		mov [ebx+0x14*8], eax
		mov [ebx+0x14*8+4] , edx
		;准备开放中断
		mov word [pidt],256*8-1            ;IDT的界限
		mov dword [pidt+2],idt_linear_address
		lidt [pidt]                        ;加载中断描述符表寄存器IDTR

	\end{lstlisting}

		\subsubsection{初始化8259A}
		\lstset{language=={[x86nasm]Assembler}}
	\begin{lstlisting}[caption={初始化8259A},tabsize=4,basicstyle=\footnotesize,captionpos=b]
		;设置8259A中断控制器
		mov al,0x11
		out 0x20,al                        ;ICW1:边沿触发/级联方式
		mov al,0x20
		out 0x21,al                        ;ICW2:起始中断向量
		mov al,0x04
		out 0x21,al                        ;ICW3:从片级联到IR2
		mov al,0x01
		out 0x21,al                        ;ICW4:非总线缓冲,全嵌套,正常EOI


		mov al,0x11
		out 0xa0,al                        ;ICW1:边沿触发/级联方式
		mov al,0x70
		out 0xa1,al                        ;ICW2:起始中断向量
		mov al,0x04
		out 0xa1,al                        ;ICW3:从片级联到IR2
		mov al,0x01
		out 0xa1,al                        ;ICW4:非总线缓冲,全嵌套,正常EOI


		;设置和时钟中断相关的硬件 
		mov al,0x0b                        ;RTC寄存器B
		or al,0x80                         ;阻断NMI
		out 0x70,al
		mov al,0x12                        ;设置寄存器B,禁止周期性中断,开放更
		out 0x71,al                        ;新结束后中断,BCD码,24小时制

		in al,0xa1                         ;读8259从片的IMR寄存器
		and al,0xfe                        ;清除bit 0(此位连接RTC)
		out 0xa1,al                        ;写回此寄存器

		mov al,0x0c
		out 0x70,al
		in al,0x71                         ;读RTC寄存器C,复位未决的中断状态

		sti                                ;开放硬件中断

	\end{lstlisting}

		\subsubsection{中断处理例程编写}
		\lstset{language=={[x86nasm]Assembler}}
	\begin{lstlisting}[caption={时钟中断处理},tabsize=4,basicstyle=\footnotesize,captionpos=b]
		rtm_0x70_interrupt_handle:                  ;实时时钟中断处理过程

		pushad

		mov al,0x20                        ;中断结束命令EOI
		out 0xa0,al                        ;向8259A从片发送
		out 0x20,al                        ;向8259A主片发送

		mov al,0x0c                        ;寄存器C的索引。且开放NMI
		out 0x70,al
		in al,0x71                         ;读一下RTC的寄存器C,否则只发生一次中断
										   ;此处不考虑闹钟和周期性中断的情况
		;转动风火轮 ,并在右下角显示
		
		xor ebx, ebx
		mov bx , [curcyc]
	  ; shr bx , 10
		and bx , 0x3
		add ebx , message_cyc
		mov cl , [ebx]
		mov ch , 0x7
		mov [VideoSite+0x0f9e],cx        ; (24*80+79)*2

		mov bx , [curcyc]
		inc bx
		mov [curcyc], bx

	  
	  mov ebx, timestrLim ; timestr+len-1, the last is '\0'
	  mov byte [ebx],0
						   ;      显示当前时间
						   ;      按照秒、分、时的顺序从后往前,从低到高位构造时间字符串

	  xor al, al                  
	  or al, 0x80
	  out 0x70, al
	  in al, 0x71
	  mov cl, al
	  and cl, 0x0f
	  add cl, '0'
	  dec ebx
	  mov [ebx], cl
	  shr al, 4
	  add al, '0'
	  dec ebx
	  mov [ebx], al
	  dec ebx
	  mov byte [ebx], ':'

	  mov al, 2
	  or al, 0x80
	  out 0x70, al
	  in al, 0x71
	  mov cl, al
	  and cl, 0x0f
	  add cl, '0'
	  dec ebx
	  mov [ebx], cl
	  shr al, 4
	  add al, '0'
	  dec ebx
	  mov [ebx], al
	  dec ebx
	  mov byte [ebx], ':'

	  mov al, 4
	  or al, 0x80
	  out 0x70, al
	  in al, 0x71
	  mov cl, al
	  and cl, 0x0f
	  add cl, '0'
	  dec ebx
	  mov [ebx], cl
	  shr al, 4
	  add al, '0'
	  dec ebx
	  mov [ebx], al

	  push ebx
	  mov ax , 0x7c6 ;     显示位置定位(24, 70)
	  shl eax, 16
	  mov ax , 0x2
	  push eax
	  call flat_4gb_code_seg_sel:simple_puts
	  add esp ,8
		popad

		iretd


	\end{lstlisting}

\lstset{language=={[x86nasm]Assembler}}
	\begin{lstlisting}[caption={键盘中断处理例程},tabsize=4,basicstyle=\footnotesize,captionpos=b]
		keyboard_interrupt_handle:         ;键盘中断处理例程
		;通过判断Scan Set 1 code的最高位,判断这次中断是按下还是弹起
pushad
mov al, 0x20
out 0xa0, al
out 0x20, al

in al, 0x60                 ; 一定要把端口里的数给读出来,不然下次中断不会被触发

mov ch, al
xor ch , 0x80
shr ch , 7                  ; 最高位为0时按下,此时颜色代码为0x4
		; 最高位为1时弹起,此时颜色代码0x0
shl ch , 2
mov cl, 'O'
mov [VideoSite+1998], cx    ; (12*8+39)*2
mov cl, 'u'
mov [VideoSite+2000], cx
mov cl, 'c'
mov [VideoSite+2002], cx
mov cl, 'h'
mov [VideoSite+2004], cx
mov cl, '!'
mov [VideoSite+2006], cx

popad
iretd
	\end{lstlisting}

\lstset{language=={[x86nasm]Assembler}}
	\begin{lstlisting}[caption={四个软中断例程},tabsize=4,basicstyle=\footnotesize,captionpos=b]
personal1_interrupt_handle:               ; 第一个自定义中断例程, 放在0x11处, 用于显示
	pushad
	mov al, 0x20                       ; 发送EOI
	out 0xa0, al
	out 0x20, al
	
	push id_info
	xor eax, eax
	mov ax, 1*80+0                     ; 先压入字符串地址,再压入坐标颜色
	shl eax, 16
	mov ax, 0x09
	push eax
	call flat_4gb_code_seg_sel:simple_puts
	add esp , 8

	popad
	iretd
personal2_interrupt_handle:               ; 第二个中断历程,在程序起始时持续显示弹跳小球
	push eax
	mov al, 0x20
	out 0xa0, al
	out 0x20, al


	push 0x0             
	push 0x0             ; BaseX Y     该历程只需要压入弹跳框的左上角基地址
	call flat_4gb_code_seg_sel:block_stone
	add esp, 8
	pop eax
	iretd
	personal3_interrupt_handle:               ;中断例程3 4 同2,改换基地址再运行几次
	push eax
	mov al, 0x20
	out 0xa0, al
	out 0x20, al

	push 0             
	push 40             ; BaseX Y
	call flat_4gb_code_seg_sel:block_stone
	add esp, 8
	pop eax
	iretd
personal4_interrupt_handle:
			push eax
	mov al, 0x20
	out 0xa0, al
	out 0x20, al

	push 12             
	push 40             ; BaseX Y
	call flat_4gb_code_seg_sel:block_stone
	add esp, 8
	pop eax
	iretd
;---------------
\end{lstlisting}

\lstset{language=={[x86nasm]Assembler}}
\begin{lstlisting}[caption={stone\_v3},tabsize=4,basicstyle=\footnotesize,captionpos=b]
	block_stone:
	pushad
	mov eax, [esp+44]
	mov [BaseX], al
	mov eax, [esp+40]
	mov [BaseY], al
	mov ecx , ShowTime                 ; 限定运动次数
										; 以下过程同实验一
	.show:
			push ecx

			mov eax, 0x0
			mov ebx, 0x0
			mov ecx, 0x0
			mov edx, 0x0
			mov al, [posx]
			mov cl, [BaseX]
			add al, cl
			mov bl, [posy]

			mov cl, [BaseY]
			add bl, cl
			mov cx, 0x50
			mul cx


			add ax, bx
			shl eax, 1
			mov ebx, VideoSite
			add ebx, eax

			mov byte [ebx], '*'
			mov cl, [esp]
			and cl, 0x7
			inc cl
			mov byte [ebx+0x1], cl

			mov ecx, [delay]
	.sleeploop:
			loop .sleeploop

			mov byte[ebx],0x0
			mov byte[ebx+0x1],0x0
	.slide:
			mov dl, [posx]
			mov dh, [posy]
			mov al, [dir]
			xor ebx, ebx
			mov bl, al
			mov al, [delx+ebx]
			mov ah, [dely+ebx]

			add dl, al
			add dh, ah
; add bl, '0'
; mov [VideoSite+4], bl
; mov byte [VideoSite+5], 0x07
; sub bl, '0'

; add al, '0'
; mov [VideoSite+6], al
; mov byte [VideoSite+7], 0x07
; sub al, '0'
			mov cl, bl
			cmp dl, 0xff
			jne .Endjudge1
					xor cl, 0x02
					mov [dir], cl
					jmp near .slide
			.Endjudge1:

			cmp dl, LimX
			jne .Endjudge2
					xor cl, 0x02
					mov [dir], cl
					jmp near .slide
			.Endjudge2:

			cmp dh, 0xff
			jne .Endjudge3
					xor cl, 0x01
					mov [dir], cl
					jmp near .slide
			.Endjudge3:

			cmp dh, LimY
			jne .Endjudge4
					xor cl, 0x01
					mov [dir], cl
					jmp near .slide
			.Endjudge4:

			mov [dir],cl
			mov [posx], dl
			mov [posy], dh

			pop ecx
			dec ecx
			cmp ecx , 0x0
			jne .show
	
	popad
	retf
	\end{lstlisting}
\lstset{language=={[x86nasm]Assembler}}
	\begin{lstlisting}[caption={simple print string},tabsize=4,basicstyle=\footnotesize,captionpos=b]
simple_puts:
pushad               ; 简单的输出字符串,不涉及光标移动
						; arg1 is string pointer
						; arg2 的低16位表示颜色,高16为表示显示的启示位置,即x*80+y (col,xy)

mov ebx , [esp+0x28] ;from 40
xor eax , eax
mov ax  , bx
shr ebx , 15         ; shr 16  ,, shl 1

mov ebp , [esp+0x2c]  ; from 44
.enumchar:
		mov cl,[ebp]
		cmp cl, 0x0   ; 字符串默认以0结尾,
		je .endenum
		mov [VideoSite+ebx], cl
		inc ebx
		mov [VideoSite+ebx], al
		inc ebx
		inc ebp
		jmp .enumchar
.endenum:
popad
retf
	\end{lstlisting}

	\subsection{user.asm}
		\subsubsection{软中断调用}
		\lstset{language=={[x86nasm]Assembler}}
	\begin{lstlisting}[caption={user0 asm},tabsize=4,basicstyle=\footnotesize,captionpos=b]
[bits 32]
	user0_length dd user0_end-user0_start
	user0_entry  dd user0_start
[section user0 vstart=0x80040500]
user0_start:
	int 0x12                                ; 依次调用自定义的软中断
	int 0x13
	int 0x14
	int 0x11

	mov cl, '#'
	mov ch, 0x07
userloop:
	mov ebx, 0x800b8004
	mov [ebx],cx                       ; 主过程不断反色地显示一个‘#’字符,
										; 观察其与时钟中断的并行程度
	xor ch, 0x1
	jmp userloop
user0_end:
	\end{lstlisting}
	
			
% 	\begin{enumerate}
% 		\item 配置bochs的指定映像文件
		
% \lstset{language=={[x86nasm]Assembler}}
% 	\begin{lstlisting}[caption={bochsrc},tabsize=4,basicstyle=\footnotesize,captionpos=b]
% 	\end{lstlisting}
	
	
		
		
% 	\end{enumerate}

	% \begin{figure}[htbp]
	% 	\centering
	% 	\includegraphics[width=10cm]{}
	% 	\bottomcaption{}
	% \end{figure}

\subsection{疑难问题解决}
	
	\begin{itemize}			
		\item 1
	\end{itemize}


\section{实验总结}

\bibliographystyle{plain}
\bibliography{ref}

\clearpage



\end{document}
